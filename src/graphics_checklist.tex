% To build this file:
%
%   latexmk -lualatex --shell-escape graphics_checklist.tex
%

\documentclass{article}

\usepackage[l2tabu,orthodox]{nag}

% ----- Type & Spacing (==
% Set English hyphenation rules.
\usepackage{polyglossia}
\setdefaultlanguage{english}

% Font selection for XeTex and LuaLatex.
\usepackage{fontspec}
\defaultfontfeatures{Ligatures=TeX}

% Adjust micro-typography.
\usepackage{microtype}
\frenchspacing

% Place all subscripts at the same height.
\usepackage{subdepth}

% Fix footnote kerning.
\usepackage{fnpct}

% 1-inch margins.
\usepackage[margin=1in]{geometry}

% Use blank lines instead of indentation to mark paragraphs.
\usepackage{parskip}
% ==)

% ----- Page Design (==
\usepackage{titlesec}
\titleformat{\section}[hang]{\sffamily\large\bfseries}{}{0em}{}
\titlespacing{\section}{0em}{0.5ex}{0.5ex}

\titleformat{\subsection}[hang]{\sffamily\normalsize\bfseries}{}{0em}{}
\titlespacing{\subsection}{0em}{0.5ex}{0.5ex}

\usepackage{enumitem}

\newlist{checklist}{itemize}{2}
\setlist[checklist]{label=$\square$}
% ==)

% ----- Links & Color (==
\usepackage{hyperref}
% ==)

% ----- Citations (==
\usepackage[
  backend = biber
  , firstinits = false
  , style = authoryear
]{biblatex}
\renewcommand*{\bibnamedash}{------. }
\setlength{\bibitemsep}{\parskip}
\setlength{\bibhang}{2em}

\addbibresource{references.bib}
% ==)

% ----- Figures (==
\usepackage{longtable}
\usepackage{booktabs}

% Highlight code with minted.
\usepackage[newfloat, cachedir=.minted]{minted}

% Use old-style numerals for line numbering.
\renewcommand{\theFancyVerbLine}{
  \sffamily{\scriptsize\oldstylenums{\arabic{FancyVerbLine}}}
}

%\definecolor{mintedbg}{rgb}{0.95, 0.95, 0.95}
\setminted{
  %resetmargins = true
  xleftmargin = 4em
  , xrightmargin = 2em
  , tabsize = 4
  % ---
  %, bgcolor = mintedbg
  %, frame = leftline
  %, framesep = 0.75em
  % ---
  , mathescape = true
  , python3 = true
  , numbers = left
  , stepnumber = 1
  , numbersep = 1em
  , showspaces = false
  , escapeinside = ~~
}

\newcommand{\code}[1]{\mintinline{text}{#1}}
% ==)

% ----- Notation (==
\usepackage{amsmath}
\usepackage{amssymb}
% ==)

\title{Graphics Checklist}
\author{STA 141A}
\date{}

\begin{document}
\maketitle{}

First, make sure you've chosen an appropriate graphic. The table below has
\textit{suggestions}. Sometimes other graphics may be more appropriate.

\begin{table}[h!]
\centering
\begin{tabular}{rrl}
First Feature & Second Feature & Plot                          \\
\midrule
categorical   &                & bar, dot                      \\
categorical   & categorical    & bar, dot, mosaic              \\
numerical     &                & box, density, histogram       \\
numerical     & categorical    & box, density                  \\
numerical     & numerical      & line, scatter, smooth scatter \\
\end{tabular}
\end{table}

Next, go through this checklist with each graphic you plan to use:

\begin{checklist}
\item \textbf{Does the graphic convey important information?} Don't include
  graphics that are uninformative or redundant.
\item \textbf{Title?} Make sure the title explains what the graphic shows.
\item \textbf{Axis labels?} Label the axes in plain language (no variable
  names!).
\item \textbf{Axis units?} Label the axes with units (inches, dollars, etc).
\item \textbf{Legend?} Any graphic that shows two or more categories coded by
  style or color must include a legend.
\item \textbf{Appropriate scales and limits?} Make sure the scales and limits
  of the axes do not lead people to incorrect conclusions. For side-by-side
  graphics or graphics that viewers will compare, use identical scales and
  limits.
\item \textbf{No overplotting?} Scatter plots where many plot points overlap
  hide the actual patterns in the data. Make the points smaller or use a
  two-dimensional density plot (a smooth scatter plot) instead.
\item \textbf{No more than 5 lines?} Line plots with more than 5 lines risk
  becoming hard-to-read ``spaghetti'' plots. Generally a line plot with more
  than 5 lines should be split into multiple plots with fewer lines. If the
  x-axis is discrete, consider using a heat map instead.
\item \textbf{Should it be a dot plot?} Pie plots are hard to read and bar
  plots don't use space efficiently \autocite{cleveland90,heer10}. Generally a
  dot plot is a better choice.
\item \textbf{Print safe?} Design graphics to be legible in black \& white.
  Color is great, but use point and line styles to distinguish groups in
  addition to color. Also try to choose colors that are accessible to
  colorblind people. The \code{RColorBrewer} and \code{viridis} packages can
  help with choosing colors.
\end{checklist}

\end{document}
